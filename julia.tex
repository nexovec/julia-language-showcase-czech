\documentclass{article}

\usepackage[czech]{babel}
\usepackage[T1]{fontenc}
\usepackage{phaistos}
\usepackage{amssymb}
\usepackage{amsmath}

\title{Programovací jazyk julia}
\author{Marek Šajner}
\begin{document}
\maketitle
\tableofcontents
\begin{abstract}
    Programovací jazyk julia se prezentuje jako alternativa k jazykům používaných v akademické sféře a numerických výpočtech,
    jako je python, R a Matlab.
    V této práci se budeme zabývat jeho výhodami a nevýhodami, jak sama o sobě, tak vzhledem k jiným programovacím jazykům.
\end{abstract}
\newpage
\section{Nastavení prostředí}
\subsection{Instalace}
\subsection{První program}
\subsection{Instalace balíčků}
\subsection{Izolované prostředí a editor}
\section{Funkcionalita jazyka}
\subsection{Syntax}
\subsection{Interoperabilita}
\subsection{Nové paradigma - multiple-dispatch}
\section{Knihovny jazyka}
\section{Testování}
\subsection{Výkonnostní testy}
\subsection{Systémové nároky}
\subsection{Dos and don'ts}
\section{Závěr}
\end{document}
\documentclass{article}

\usepackage[czech]{babel}
\usepackage[T1]{fontenc}
\usepackage{phaistos}
\usepackage{amssymb}
\usepackage{amsmath}
\usepackage{listings}
\usepackage{color}

\definecolor{dkgreen}{rgb}{0,0.6,0}
\definecolor{gray}{rgb}{0.5,0.5,0.5}
\definecolor{mauve}{rgb}{0.58,0,0.82}

\lstset{frame=tb,
  language=python,
  aboveskip=3mm,
  belowskip=3mm,
  showstringspaces=false,
  columns=flexible,
  basicstyle={\small\ttfamily},
  numbers=none,
  numberstyle=\tiny\color{gray},
  keywordstyle=\color{blue},
  commentstyle=\color{dkgreen},
  stringstyle=\color{mauve},
  breaklines=true,
  breakatwhitespace=true,
  tabsize=3
}

\title{Programovací jazyk julia}
\author{Marek Šajner}
\begin{document}
\maketitle
\tableofcontents
\begin{abstract}
    Programovací jazyk julia se prezentuje jako alternativa k jazykům používaných v akademické sféře a numerických výpočtech,
    jako je python, R a Matlab.
    V této práci ho budu posuzovat z hlediska výkonnosti, uživatelského prostředí a knihoven.
\end{abstract}
\newpage
\section{Nastavení prostředí}
\subsection{První program}
Oficiální implementaci julie lze stáhnout z oficiálních stránek(https://julialang.org/downloads/).
Po nainstalování je možné spustit julii z příkazové řádky pomocí příkazu julia.

Tento mód chodu interpreteru julia se označuje jako REPL (Read-Eval-Print-Loop),
a v tomto módu je možné zadávat příkazy a ihned vidět jejich výstup.
Pro spuštění skriptu ze souboru je možné použít příkaz
\begin{lstlisting}
    include("<cesta k souboru>"),
\end{lstlisting}
nebo lze soubor spustit přímo z příkazové řádky pomocí příkazu julia <cesta k souboru>.

Pro základní nápovědu k použití REPL je možné použít příkaz ? nebo help() uvnitř REPLu.
Pro nápovědu k rozhraní souboru z příkazové řádky(CLI rozhraní) je možné použít příkaz julia --help.
Dokumentace k julii je v době psaní tohoto dokumentu dostupná na oficiálních stránkách(https://docs.julialang.org/en/v1/),
a tato dokumetace pokývá jak syntax jazyka, tak i jeho knihovny.

\subsection{Instalace balíčků}
Management balíčků v julii je realizován pomocí balíčku Pkg, který je součástí standardní instalace.
Můj oblíbený způsob instalace balíčků je přes REPL, kde je možné napsat znak ], který přepne REPL do balíčkového módu.
Zde lze nainstalovat balíček příkazem add. Balíčky typicky vyhledávám online na stránkách https://juliapackages.com/.

Jeden z populárních balíčků je Plots, který slouží k vykreslování grafů, a ten teď nainstalujeme příkazem add Plots.
Zpět do REPLu se dostaneme příkazem backspace.

Instalovat balíčky lze pochopitelně i uvnitř skriptu, v našem případě příkazem
import Pkg
Pkg.add("Plots")

Oba způsoby jsou plnohodnotné a volají stejný příkaz. Stejně to je i s ostatními příkazy uvnitř Pkg.

\subsection{Izolované prostředí a editor}

Julia poskytuje líbivý způsob pro vytvoření izolovaného prostředí pro jednotlivé projekty.
\begin{lstlisting}
    import Pkg
\end{lstlisting}
a alternativně

% Pkg.activate("<libovolný název prostředí>")

% Pkg.activate(<cesta k prostředí>)
\begin{lstlisting}
\end{lstlisting}

Toto zabrání instalaci balíčků do globálního prostředí,
čímž se eliminují konflikty mezi verzemi balíčků v různých projektech,
a také to usnadňuje management ballíčků, jelikož prostředí lze lehce odstranit,
a následně smazat nepoužívané balíčky.

Zároveň je možné vyexportovat požadavky projektů, což je užitečné pro kolaborativní práci.
Jedná se o formu aktivní dokumentace projektu, což je samo o sobě dobré.
% tady se dá bavit o manifestech

\section{Funkcionalita jazyka}
\subsection{Syntax}
Následující úryvek kódu ilustruje syntax julie.
\subsection{Interoperabilita}
\subsection{Nové paradigma - multiple-dispatch}
\section{Knihovny jazyka}
\section{Testování}
\subsection{Výkonnostní testy}
\subsection{Systémové nároky}
\subsection{Dos and don'ts}
\section{Závěr}
\end{document}